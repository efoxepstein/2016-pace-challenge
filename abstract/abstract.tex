\documentclass{article}

\usepackage{amsthm,amsmath,amssymb,amsfonts,xspace,color,enumerate,graphicx,url,xfrac}

\newcommand{\CC}{C\nolinebreak\hspace{-.05em}\raisebox{.4ex}{\tiny\bf +}\nolinebreak\hspace{-.10em}\raisebox{.4ex}{\tiny\bf +}}

\title{Minimum Degree Heuristic}
\author{Eli Fox-Epstein}

\begin{document}
\maketitle

This abstract corresponds to the submission for the 2016 PACE Challenge. We are
entering Track A and provides a serial, heuristic implementation.  Source code
is available at: \url{https://github.com/elitheeli/2016-pace-challenge}.

This entry is provided as a baseline, implementing simple strategy: the Minimum
Degree Heuristic~\cite{markowitz1957elimination}. In the time given, this
implementation runs the heuristic repeatedly, maintaining the best
decomposition.

{\bf Implementation Details.}
This implementation is written in \CC{}11. Tree decompositions are represented
by two vectors: one is a vector of bags (each of which is a vector of vertex
IDs) and one stores parent-child relationships.

The graph is represented using an adjacency matrix (compactly represented as a
{\tt std::vector<bool>}), as well as a adjacency list for each vertex (using a
{\tt std::vector}) to support efficient neighborhood iteration and a vector
tracking vertex degrees.  At any given time, the adjacency lists contain a
\emph{superset} of the neighbors of a vertex: on iteration through a
neighborhood, each potential neighbor is verified with the matrix and discarded
if it is not a true neighbor.

The algorithm maintains a priority queue on the vertices, keyed by degree plus
a small amount of random noise. The noise ensures a random choice among the
vertices with minimum degree. Three vectors are maintained: one to store the
order in which vertices are selected, one mapping a vertex to whether or not it
has been selected, and one tracking the degree of a vertex when enqueued.
Initially, all vertices are enqueued.  When processing a vertex~$v$, if~$v$ has
already been seen, it is simply discarded.  If its degree is less than its
current degree, it is re-enqueued with the correct degree.  Finally, if its
degree is correct, it is added to the ordering, marked as visited, and its
neighbors are connected into a clique and then disconnected from~$v$. Some
neighbors may be re-enqueued if their degrees decreased by losing their edge
to~$v$.

\bibliographystyle{plain}
\bibliography{bibliography}

\end{document}
